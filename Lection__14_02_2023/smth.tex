\documentclass[a4paper,12pt]{article}
\usepackage{cmap}			
\usepackage[T2A]{fontenc}	
\usepackage[utf8]{inputenc}	
\usepackage[english,russian]{babel}	
% \usepackage{graphicx}
% \usepackage{caption}
% \usepackage{amsmath, amsfonts}
% \usepackage{amsmath}
% \usepackage{empheq}
% \usepackage{enumitem}
% \usepackage{tikz}
% \usepackage{keycommand}
% \usepackage{cases}
% \usepackage{xcolor,colortbl}
% \usepackage{amssymb}
% \usepackage{pdfpages}

\usepackage[left=2.5cm, top=2cm, right=1cm, bottom=20mm, nohead, nofoot]{geometry}

\begin{document}
    Опозиция $\leftrightarrow$ психология

    \begin{enumerate}
        \item Мышлене - объект(предметность). Т.е. существует сферически йконь в вакуме. Фиксация границ (ограничение).
        \item Возмём ч-ка, он может быть обращён к объективной(то что данно, здесь нет мышление то, что является основой) или субьективной(то что предпологается, здесь есть мышление и оно здесь звено) реальности.
        \item 
    \end{enumerate}
    Где проявляется мышление.
    \begin{enumerate}
        \item Сенсорное познание -- Обектвный мир дан в ощущениях, идёт конструирование ощущений.
        % синкретические мышления
        \item Перцептивный слой (первичные мыслительные конструкции)(образные проявления) -- массив чего-то, что станет существенным. -- картина мира (структурированный массив обьектов с неким содержанием)
        \item Осмысленный слой -- слой значений и смыслов.
        \item 
    \end{enumerate}
    Опозици по содержанию
    \begin{enumerate}
        \item Мышление - снятие неорпределённости и создание ясности. 
        \begin{itemize}
            \item Решение задач вопрос - ответ
            \item Творчество матерьял - новое
            \item Модель принятия решений
        \end{itemize}
        \item 
    \end{enumerate}

\end{document}